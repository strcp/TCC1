\section{Introdução}
Este trabalho demonstra primeiramente os conceitos e utilizações do paradigma de
\textit{Internet of Things} e suas derivações. Através de um breve estudo do estado
da arte das plataformas, ambientes e aplicações vigentes neste contexto, procuramos
demonstrar o espaço que há para inovações e experimentos na área e a necessidade
dos novos produtos de se adaptarem a esta nova realidade da tecnologia.

Por fim, é proposta uma integração de dois produtos já existentes, o sistema
operacional de tempo real HellfireOS e a plataforma de serviços para \textit{Internet of Things}
chamada COMPaaS, ambos desenvolvidos pelo Grupo de Sistema Embarcados da Pontifícia Universidade
Católica do Rio Grande do Sul, criando uma solução completa e apta a integrar o estado da
arte no conceito de IoT.
