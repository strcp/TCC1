\section{Introdução}
\subsection{História dos Sistemas de Tempo Real}
lorem ipsum dolor sit amet lorem ipsum dolor sit amet consetetur

\subsection{Tempo}
\paragraph{}
O tempo possui diferentes interpretações, especialmente na área computacional,
podendo ser analisado em diferentes aspectos. Sendo seu entendimento de
fundamental importância para a compreensão e análise dos sistemas de tempo
real, abordaremos a comparação de algumas dessas definições a seguir.~\cite{FARINESSIS00}

\begin{itemize}
\item \textbf{Tempo na Execução}

É o prerído que uma tarefa necessita para ser executada. Assim como recursos
físicos ou lógicos, é interpretado como um recurso que pode ser consumido
por uma processo na sua execução.

\item \textbf{Tempo na Programação}

lorem ipsum dolor sit amet consetetur sadipscing elitr sed diam nonumy

\item \textbf{Tempo Lógico}

lorem ipsum dolor sit amet consetetur sadipscing elitr sed diam nonumy

\item \textbf{Tempo Físico}

lorem ipsum dolor sit amet consetetur sadipscing elitr sed diam nonumy

\item \textbf{Tempo Denso}

lorem ipsum dolor sit amet consetetur sadipscing elitr sed diam nonumy

\item \textbf{Tempo Discreto}

lorem ipsum dolor sit amet consetetur sadipscing elitr sed diam nonumy

\item \textbf{Tempo Global}

lorem ipsum dolor sit amet consetetur sadipscing elitr sed diam nonumy

\item \textbf{Tempo Local}

lorem ipsum dolor sit amet consetetur sadipscing elitr sed diam nonumy

\item \textbf{Tempo Absoluto}

lorem ipsum dolor sit amet consetetur sadipscing elitr sed diam nonumy

\item \textbf{Tempo Relativo}

lorem ipsum dolor sit amet consetetur sadipscing elitr sed diam nonumy

\end{itemize}

\subsection{Definição de um Sistema de Tempo Real}
\paragraph{}
Um sistema de tempo real, ao contrário do que se costuma pensar, não tem
como objetivo uma execução necessariamente rápida, mas sim, previsível.
Deste tipo de sistema, podemos encontrar duas características fundamentais:

\begin{itemize}
\item Um sistema de tempo real precisa produzir resultados computacionais corretos
(chamados de corretude logica ou funcional).
\item Os resultados computacionalmente corretos obtidos precisam ser concluídos em
um período de tempo pré-definido, caracterizando a previsibilidade na execução da tarefa.
\end{itemize}

\paragraph{}
Sistemas de tempo real são definidos como sistemas nos quais a corretude nos
resultados de execução de maneira geral são dependentes tanto da corretude
lógica como da previsibilidade do tempo de execução, sendo assim, a previsibilidade
do tempo de execução é ao menos tão importante quanto a corretudo funcional
nesse tipo de sistemas.~\cite{Li:2003:RCE:829584}

\subsubsection{Aplicações}
\paragraph{}
Os sistemas de tempo real são utilizados para atender à tarefas que possum algum tipo
de restrição temporal em sua execução. Esse tipo de necessidade está muito presente no
dia-a-dia, e são aplicados aos mais diversos tipos de tarefas, desde controladores de
leitores de DVDs, elevadores até freios de carro, controladores de mísseis e piloto automático
de aeronaves.

\subsubsection{Problemas clássicos de tempo real}
\paragraph{}
lorem ipsum dolor sit amet consetetur sadipscing elitr sed diam nonumy


\subsection{Tipos de Sistema de Tempo Real}
\paragraph{}
Como abordado na seção (referenciar seção Definição de um Sistema de Tempo Real),
para que um sistemas de tempo real obtenha uma execução de tarefa correta,
é necessário terminar essa execução em um período pré-definido, chamado de \textit{deadline}.
Portanto, é possível afirmar que este tipo de sistema, por possuir essa restrição temporal,
é regido pelos \textit{deadlines} de suas tarefas.
\paragraph{}
Devido a importância dos \textit{deadlines}, esse tipo de sistema pode ser classificado como
críticos ou não críticos, baseado na tolerância de \textit{deadlines} perdidos, a utilidade
dos resultados computados após a perda do \textit{deadline} e a severidade da penalidade em
perder um prazo de execução.

\subsubsection{Sistemas de Tempo Real Críticos}
\paragraph{}
Os sistemas de tempo real chamados de críticos são aqueles que possuem uma restrição severa
de prazo na execução de tarefas, ou seja, sua tolerancia em perder prazos é muito pequena
ou nenhuma. Em muitos desses sistemas, as informações computadas fora do prazo são consideradas
inúteis, caracterizando uma penalidade grave em perder o prazo de execução.
\paragraph{}
Um sistema de tempo real crítico é um sistema que precisa executar suas tarefas dentro do prazo
pré-definido com uma tolerância muito próxima de zero. Os prazos devem ser atendidos, ou os resultados
são catastróficos. O custo da perda de prazos na execução possum um custo muito alto, podendo inclusive
involver vidas humanas. Os resultados computados após o prazo pré-definido possuem uma utilidade próxima
de zero ou possuem um grande grau de depreciação com a decorrência do tempo após o prazo.~\cite{Li:2003:RCE:829584}
\paragraph{}
Alguns exemplos de sistemas de tempo real considerados críticos são:
\begin{itemize}
\item Sistema de navegação de aeronaves.
\item Freios de carro (ABS).
\item Marca-passo.
\item Controladores de mísseis.
\end{itemize}

\subsubsection{Sistemas de Tempo Real Não Críticos}
\paragraph{}
Os sistemas de tempo real considerados não críticos são aqueles aonde uma perda de prazo
resulta em uma penalidade leve, como uma distorção em uma música sendo lida de um CD, perda
de alguns frames em um vídeo entre outras consequências de baixa criticidade.


\subsection{Tipos de escalonamentos}
lorem ipsum dolor sit amet consetetur sadipscing elitr sed diam nonumy

\subsubsection{Rate monotonic scheduling}
lorem ipsum dolor sit amet consetetur sadipscing elitr sed diam nonumy

\subsubsection{Round-Robin}
lorem ipsum dolor sit amet consetetur sadipscing elitr sed diam nonumy

\subsubsection{Fixed-Priority}
lorem ipsum dolor sit amet consetetur sadipscing elitr sed diam nonumy

\subsubsection{Critical section preemptive scheduling}
lorem ipsum dolor sit amet consetetur sadipscing elitr sed diam nonumy

\subsubsection{Static time scheduling}
lorem ipsum dolor sit amet consetetur sadipscing elitr sed diam nonumy

\subsubsection{Earliest Deadline First}
lorem ipsum dolor sit amet consetetur sadipscing elitr sed diam nonumy

\subsubsection{Cooperative scheduling}
lorem ipsum dolor sit amet consetetur sadipscing elitr sed diam nonumy


\subsection{Testes}
lorem ipsum dolor sit amet consetetur sadipscing elitr sed diam nonumy

\subsubsection{Teste de tarefas}
lorem ipsum dolor sit amet consetetur sadipscing elitr sed diam nonumy

\subsubsection{Teste de comportamento}
lorem ipsum dolor sit amet consetetur sadipscing elitr sed diam nonumy

\subsubsection{Teste intertarefa}
lorem ipsum dolor sit amet consetetur sadipscing elitr sed diam nonumy

\subsubsection{Teste do sistema}
lorem ipsum dolor sit amet consetetur sadipscing elitr sed diam nonumy
