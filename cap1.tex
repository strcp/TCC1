\section{Introdução}
\subsection{História dos Sistemas de Tempo Real}

\subsection{Tempo}
\paragraph{}
O tempo possui diferentes interpretações, especialmente na área computacional,
podendo ser analisado em diferentes aspectos. Sendo seu entendimento de
fundamental importância para a compreensão e análise dos sistemas de tempo
real, abordaremos a comparação de algumas dessas definições a seguir.~\cite{FARINESSIS00}

\begin{itemize}
\item \textbf{Tempo na Execução}

É o período que uma tarefa necessita para ser executada. Assim como recursos
físicos ou lógicos, é interpretado como um recurso que pode ser consumido
por uma processo na sua execução.

\item \textbf{Tempo na Programação}

lorem ipsum dolor sit amet consetetur sadipscing elitr sed diam nonumy

\item \textbf{Tempo Lógico}

lorem ipsum dolor sit amet consetetur sadipscing elitr sed diam nonumy

\item \textbf{Tempo Físico}

lorem ipsum dolor sit amet consetetur sadipscing elitr sed diam nonumy

\item \textbf{Tempo Denso}

lorem ipsum dolor sit amet consetetur sadipscing elitr sed diam nonumy

\item \textbf{Tempo Discreto}

lorem ipsum dolor sit amet consetetur sadipscing elitr sed diam nonumy

\item \textbf{Tempo Global}

lorem ipsum dolor sit amet consetetur sadipscing elitr sed diam nonumy

\item \textbf{Tempo Local}

lorem ipsum dolor sit amet consetetur sadipscing elitr sed diam nonumy

\item \textbf{Tempo Absoluto}

lorem ipsum dolor sit amet consetetur sadipscing elitr sed diam nonumy

\item \textbf{Tempo Relativo}

lorem ipsum dolor sit amet consetetur sadipscing elitr sed diam nonumy

\end{itemize}

\subsection{Definição de um Sistema de Tempo Real}
\paragraph{}
Um sistema de tempo real, ao contrário do que se costuma pensar, não tem
como objetivo uma execução necessariamente rápida, mas sim, previsível.
Deste tipo de sistema, podemos encontrar duas características fundamentais:

\begin{itemize}
\item Um sistema de tempo real precisa produzir resultados computacionais corretos
(chamados de corretude logica ou funcional).
\item Os resultados computacionalmente corretos obtidos precisam ser concluídos em
um período de tempo pré-definido, caracterizando a previsibilidade na execução da tarefa.
\end{itemize}

\paragraph{}
Sistemas de tempo real são definidos como sistemas nos quais a corretude nos
resultados de execução de maneira geral são dependentes tanto da corretude
lógica como da previsibilidade do tempo de execução, sendo assim, a previsibilidade
do tempo de execução é ao menos tão importante quanto a corretude funcional
nesse tipo de sistemas.~\cite{Li:2003:RCE:829584}

\subsubsection{Aplicações}
\paragraph{}
Os sistemas de tempo real são utilizados para atender à tarefas que possuem algum tipo
de restrição temporal em sua execução. Esse tipo de necessidade está muito presente no
dia-a-dia, e são aplicados aos mais diversos tipos de tarefas, desde controladores de
leitores de DVDs, elevadores até freios de carro, controladores de mísseis e piloto automático
de aeronaves.

\subsubsection{Problemas Clássicos de Tempo Real}
\paragraph{}
lorem ipsum dolor sit amet consetetur sadipscing elitr sed diam nonumy


\subsection{Tipos de Sistema de Tempo Real}
\paragraph{}
Como abordado na seção (referenciar seção Definição de um Sistema de Tempo Real),
para que um sistemas de tempo real obtenha uma execução de tarefa correta,
é necessário terminar essa execução em um período pré-definido, chamado de \textit{deadline}.
Portanto, é possível afirmar que este tipo de sistema, por possuir essa restrição temporal,
é regido pelos \textit{deadlines} de suas tarefas.
\paragraph{}
Devido a importância dos \textit{deadlines}, esse tipo de sistema pode ser classificado como
críticos ou não críticos, baseado na tolerância de \textit{deadlines} perdidos, a utilidade
dos resultados computados após a perda do \textit{deadline} e a severidade da penalidade em
perder um prazo de execução.

\subsubsection{Sistemas de Tempo Real Críticos}
\paragraph{}
Os sistemas de tempo real chamados de críticos são aqueles que possuem uma restrição severa
de prazo na execução de tarefas, ou seja, sua tolerância em perder prazos é muito pequena
ou nenhuma. Em muitos desses sistemas, as informações computadas fora do prazo são consideradas
inúteis, caracterizando uma penalidade grave em perder o prazo de execução.
\paragraph{}
Um sistema de tempo real crítico é um sistema que precisa executar suas tarefas dentro do prazo
pré-definido com uma tolerância muito próxima de zero. Os prazos devem ser atendidos, ou os resultados
são catastróficos. O custo da perda de prazos na execução possuem um custo muito alto, podendo inclusive
envolver vidas humanas. Os resultados computados após o prazo pré-definido possuem uma utilidade próxima
de zero ou possuem um grande grau de depreciação com a decorrência do tempo após o prazo.~\cite{Li:2003:RCE:829584}
\paragraph{}
Alguns exemplos de sistemas de tempo real considerados críticos são:
\begin{itemize}
\item Sistema de navegação de aeronaves.
\item Freios de carro (ABS).
\item Marca-passo.
\item Controladores de mísseis.
\end{itemize}

\subsubsection{Sistemas de Tempo Real Não Críticos}
\paragraph{}
Os sistemas de tempo real considerados não críticos são aqueles aonde uma perda de prazo
resulta em uma penalidade leve, como uma distorção em uma música sendo lida de um CD, perda
de alguns frames em um vídeo entre outras consequências de baixa criticidade.
\paragraph{}
Esses sistemas precisam atender seus prazos pré-definidos, porém com um certo grau de flexibilidade.
Os prazos podem possuir diferentes graus de tolerância, prazos balizados com tempo médio e até em avaliação
estatística dos tempos de resposta. Nesse tipo de sistema, apesar de a perda de prazos não resultar em uma
falha no sistema, os custos da perda desses prazos pode se tornar grande dependendo da proporção em que isso
ocorre.~\cite{Li:2003:RCE:829584}

\paragraph{}
Alguns exemplos de sistemas de tempo real considerados não críticos são:
\begin{itemize}
\item Sistema responsável por leitura de DVD.
\item Sistema de som em computadores pessoais.
\item Decodificador de sinal de televisão.
\end{itemize}

\subsection{Tipos de Escalonamentos}
\paragraph{}
Como todos os sistemas operacionais, os sistemas de tempo real também possuem algoritmos de escalonamento.
Devido à natureza deste tipo de sistemas, seus algoritmos para escolha de execução possuem diferenças
dos escalonadores em sistemas operacionais normais, visando, especialmente, atender aos requisitos temporais
das tarefas na linha de execução. Como os sistemas operacionais de tempo real possuem diferentes aplicações
com diferentes restrições, uma série de algoritmos de escalonamento foi desenvolvida, buscando atender de
maneira mais otimizada cada uma destas necessidades.
\paragraph{}
Esses algoritmos de escalonamento podem ser classificados como dois tipos, estáticos ou dinâmicos.
Os escalonadores estáticos são geralmente utilizado quando os sistema executa tarefas que são previamente
conhecidas, assim como seus prazos de execução e inclusive seu pior caso de execução.
\paragraph{}
Os algoritmos de escalonamento dinâmicos tomam as decisões de escalonamento em tempo de execução, não
atribuindo prioridades fixas às tarefas, podendo modificar essas prioridades dependendo dos critérios
do algoritmo utilizado.

\subsubsection{Rate Monotonic Scheduling}
\paragraph{}
O \textit{Rate Monotonic Scheduling} (RMS), é um algoritmo de escalonamento amplamente utilizado em sistemas
de tempo real. É um algoritmo considerado estático, já que as prioridades dos processos são setadas estaticamente
baseada na duração do ciclo de execução de cada tarefa, quanto menor seu ciclo, maior sua prioridade.

\subsubsection{Earliest Deadline First}
\paragraph{}
É um algoritmo de escalonamento dinâmico. Seu algoritmo consiste basicamente em avaliar a lista de processo
pendentes a cada término de processamento, dando maior prioridade para os processos que possuem seus
\textit{deadlines} mais próximo, caracterizando seu nome.

\subsubsection{Round-Robin}
\paragraph{}
Amplamente utilizado em escalonamento de pacotes de redes de computadores e sistemas operacionais sem características
de tempo real, é também utilizado em sistemas de tempo real. É um algoritmo preemptivo e dinâmico, se baseando no
conceito de fatias de tempo iguais para execução de cada processo, alocados como uma lista circular.

\subsection{Testes}
\paragraph{}
Segundo Roger S. Pressman~\cite{pre_2005}, o design de caso de testes para sistemas de tempo real pode ser dividido
em quatro passos:

\subsubsection{Teste de Tarefas}
\paragraph{}
No princípio, cada tarefa é testada individualmente com maneiras convencionais de testes estáticos, buscando encontrar
apenas erros de lógica ou sintaxe no programa. É um tipo de teste que não leva em conta especificamente as peculiaridades
dos sistemas de tempo real, não levando em conta ainda as restrições temporais impostas ao sistema e suas tarefas.

\subsubsection{Teste de Comportamento}
\paragraph{}
Utilizando os modelos de sistema desenvolvidos com ferramentas para automação de testes, é possível simular os comportamentos
de um sistema de tempo real e os impactos de eventos externos sobre esses comportamentos.

\subsubsection{Teste Intertarefa}
\paragraph{}
Partindo do pressuposto de que as tarefas estão livres de erros, sendo isso validado pelo \textit{teste de tarefas}, as tarefas são
testadas adicionando ao cenário as características de restrição temporal impostas ao sistema. Esse tipo de teste visa encontrar
erros de comunicação interprocesso, testando tarefas assíncronas com diferentes taxas e tamanhos de dados.

\subsubsection{Teste do Sistema}
\paragraph{}
Nesse ponto, software e hardware são integrados e uma larga escala de testes de sistema são feito para encontrar erros durante
a comunicação entre software e hardware.
