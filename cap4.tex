\section{Proposta}
\subsection{Proposta TCC 1}
\subsubsection{Objetivos}
Como objetivos da primeira etapa do trabalho de conclusão, buscamos, em primeiro lugar,
elevar nosso conhecimento nas áreas de sistemas de tempo real e especialmente
em métodos formais, com suas ferramentas e aplicações. Iremos também,
baseado no conhecimento de pesquisa feita no decorrer deste trabalho, modelar um sistema
nos diferentes métodos formais de especificação descritos na seção~\ref{sec:MFeSTR} para fins de
prova de conceito, possibilitando a avaliação destes com base nessas experiências de implementação,
analisando quais são os seus pontos fortes e fracos e quais ferramentas melhor se adaptam às nossas
necessidades para implementarmos a segunda etapa desse trabalho de conclusão.
\\

\paragraph{Sistema Escolhido}\mbox{} \\\\
No mundo da computação vários problemas são resolvidos sem levar em conta o tempo.
Isto decorre da natureza dos problemas obedecerem a lógica não modal, fazendo com que o
problema seja resolvido sem precisarmos atentar a aspectos temporais. Apesar disso,
existem sistemas aonde o aspecto temporal se faz presente, sistemas reativos são um exemplo disto.

Para a análise preliminar dos métodos formais de modelagem propostos, foi escolhido um problema que
já é conhecido pela literatura, exposto por Abrial, B\"{o}rger e Langmaack~\cite{opac-b1092561},
para que seja possível utilizar como exemplo de problema a ser modelado.

O propósito do \textit{Sistema de Aquecimento de Água} (SAA) é o de garantir o funcionamento,
de forma segura, do aquecedor de água. O aquecedor de água opera em segurança se o nível da água
não exceder o limite de tolerância. Além desta restrição, o tanque do sistema tem também
sua resistência à pressão, que não pode ser maior que um valor estipulado pelo fabricante.

O SAA é composto por uma série de sistemas que são necessários para o aquecimento da água e também
para o monitoramento das condições de operação. A seguir são listados os principais subsistemas:
\begin{itemize}
\item Um tanque para o aquecimento da água.
\item Dispositivo para a medição do nível da água.
\item Uma bomba d'água para o abastecimento do tanque.
\item Um dispositivo para a medição da bomba d'água.
\item Um dispositivo para medir a pressão dentro do tanque.
\item Um sistema de controle para atuar nos dispositivos.
\end{itemize}

Para criação da prova de conceito em relação aos métodos formais de modelagem de sistemas previamente
mencionada, utilizaremos um destes subsistemas, nos capacitando tanto para a escolha do método quanto
para o conhecimento das linguagens, ferramentas e técnicas utilizadas para essa finalidade de modelagem.

\subsubsection{Cronograma}
\renewcommand{\arraystretch}{1.5}
\definecolor{lightgray}{gray}{0.9}
\rowcolors{2}{white}{lightgray}

\newcolumntype{C}{>{\centering\arraybackslash}X}

\begin{tabularx}{\textwidth}{ | c | C | }
\hline
\textbf{Data} & \multicolumn{1}{c|}{\textbf{Evento}} \\
\hline
16.09.2014 & Escolha do subsistema do SSA a ser especificado para prova de conceito. \\

17.09.2014 & Escolha da ferramenta para especificação formal do sistema utilizando Statecharts. \\
17.09.2014 & Início da especificação formal do sistema utilizando Statecharts. \\
10.10.2014 & Término da especificação formal do sistema utilizando Statecharts. \\

11.10.2014 & Início da especificação formal do sistema utilizando Alloy. \\
04.11.2014 & Término da especificação formal do sistema utilizando Alloy. \\

05.11.2014 & Escolha da ferramenta para especificação formal do sistema utilizando CSP. \\
05.11.2014 & Início da especificação formal do sistema utilizando CSP. \\
20.11.2014 & Término da especificação formal do sistema utilizando CSP. \\

24.11.2014 & Entrega do Volume Final \\
\hline
\end{tabularx}

\subsubsection{Lista de Atividades}


\subsection{Proposta TCC 2}
\subsubsection{Objetivos}
Como objetivos da segunda etapa do trabalho de conclusão, temos a intenção
de aplicar o conhecimento obtido na primeira etapa do trabalho de forma prática,
além de buscar mais conhecimento na área específica de verificação formal de sistemas.
Para que possamos atingir as metas descritas acima, iremos escolher um método formal
de especificação de sistemas, baseados nas pesquisas e experiências desenvolvidas
na primeira etapa do trabalho, para especificar um sistema de tempo real de maior magnitude,
possibilitando a verificação deste através do conhecimento agregado nesta segunda etapa.

\subsubsection{Cronograma}
\renewcommand{\arraystretch}{1.5}
\definecolor{lightgray}{gray}{0.9}
\rowcolors{2}{white}{lightgray}

\newcolumntype{C}{>{\centering\arraybackslash}X}

\begin{tabularx}{\textwidth}{ | c | C | }
\hline
\textbf{Data} & \multicolumn{1}{c|}{\textbf{Evento}} \\
\hline
xx.03.2015 & Escolha do sistema de tempo real a ser especificado e verificado. \\
xx.03.2015 & Escolha da especificação formal e ferramenta a ser utilizada. \\
xx.03.2015 & Início da especificação formal do sistema. \\
xx.xx.2015 & Fim da especificação formal do sistema. \\
xx.xx.2015 & Verificação formal do sistema. \\
xx.xx.2015 & Entrega do Volume Final \\
xx.xx.2015 & Apresentação do Trabalho \\
\hline
\end{tabularx}

\subsubsection{Lista de Atividades}
