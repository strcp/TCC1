\section{Cooperative Middleware Platform as a Service}
\label{sec:COMPaaS}
Cooperative Middleware Platform as a Service (COMPaaS) é um sistema que provê aos usuários uma simples e bem
definida plataforma de serviços. Ele é composto por 3 subsistemas, a saber:  Application Level Events
Interface (ALE), Middleware Level Events Interface (MLE) e Device Level Protocol (DLE).

\subsection{Application Level Events Interface}
O ALE expõe uma interface SOAP para a comunicação com as aplicações. Ele é responsável por receber as requisições
das aplicações e encaminha-las ao MLE. As informações requisitadas são recebidas do MLE por meio de um WebSocket.

\subsection{Middleware Level Events Interface}
O MLE expõe uma interface RESTful para a comunicação com o ALE. É responsável pelo recebimento das requisições
oriundas do ALE para então encaminhar as mesma para o DLE. Os dados retornados do DLE, no formato de DataMessage
object, são então enviados ao ALE por meio de WebSocket.

\subsection{Device Level Protocol}
O DLE cria um dispositivo virtual em torno do dispositivo físico e é responsável por definir o formato do DataMessage
object e expor as funcionalidades que são realizadas pelo dispositivo físico. O DLE é o responsável por saber das
minúcias de cada dispositivo, então, para cada novo dispositivo que se deseja incluir na IoT, é necessário criar
um DLE específico para aquele dispositivo. Isto é um pouco oneroso e será um dos temas abordados neste trabalho.

\subsection{Arquitetura}
