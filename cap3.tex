\section{Métdodos Formais e Sistemas de Tempo Real}
\label{sec:MFeSTR}
As especificações formais são técnicas baseadas em matemática utilizadas para descrever um sistema,
possibilitando sua análise e verificação de propriedades.

Não são todos os formalismos que levam em conta o aspecto temporal do problema a ser modelado.
Em busca de um modelo formal que contemplasse estes requisitos temporais, foram desenvolvidos uma série
de modelos que serão abordados a seguir.

\subsection{Álgebra de Processos}
A álgebra de processos é uma abordagem para possibilitar o estudo de processos concorrentes. Utiliza linguagem algébrica
para a especificação de processos e a formulação de afirmações acerca deles, assim como utiliza o cálculo para fazer a
verificação dessas afirmações. Segundo Petri Net Algebra~\cite{books/daglib/0003970}, a álgebra de processos é a área da
Ciência da Computação que se dedica a estudar formalmente a sequência de execução.

Apesar de haver uma série de abordagens algébricas a modelagem e análise de processos concorrentes, o foco neste trabalho
será o \textit{Communicating Sequential Process} abordado a seguir.

\subsubsection{Communicating Sequential Process (CSP)}
\lipsum[1]


\subsection{Alloy}
Alloy é uma linguagem formal de especificação utilizada para expressar sistemas que possuem
estruturas e comportamentos com restrições complexas. As fundações matemáticas deste formalismo
foram muito influenciadas pela notação Z~\cite{opac-b1091336}, porém sua sintaxe foi influenciada
por linguagens como \textit{Object Constraint Language}(OCL), buscando ser mais simples para
criação de modelos de forma incremental.

A primeira versão da linguagem Alloy surgiu em 1997 e era apenas uma limitada linguagem para modelagem
de objetos. Sua evolução aconteceu mais tarde com a adição de quantificadores, polimorfismo, subtipos e
assinaturas. Essa linguagem se diferencia de boa parte das linguagens formais de especificação desenvolvidas
para verificação de modelos no sentido de que aceita a definição de modelos infinitos.

\subsubsection{Ferramentas}
\begin{itemize}
\item{Alloy Analyzer}

É uma ferramenta que analisa especificações escritas na linguagem Alloy. Como a linguagem, essa ferramenta
foi desenvolvida no \textit{Massachusetts Institute of Technology} (MIT). Uma das grandes vantagens do Alloy
Analyzer é sua capacidade de realizar a análise de modelos parciais, com isso, pode efetuar análises incrementais
dos modelos enquanto são criados e fornecer resultados instantâneos para o usuário.

\end{itemize}


\subsection{Statecharts}
Statecharts~\cite{Harel:1987:SVF:34884.34886}, é um formalismo criado por
David Harel nos anos 80 que possibilita descrever, de forma rigorosa como a
álgebra de processos, sistemas complexos de forma gráfica, possibilitando uma
visualização melhor do sistema em questão.

A grande parte da literatura que compreende o desenvolvimento de sistemas atende
apenas às necessidades de sistemas transformacionais, State charts foi criado para
atacar este déficit, atendendo aos interesses de sistemas reativos orientado a eventos.

\subsubsection{Ferramentas}
\paragraph{}
www.statecharts.org


