\section{Estudo de Caso}
No mundo da computação vários problemas são resolvidos sem levar em conta o tempo. 
Isto decorre da natureza dos problemas obedecerem a lógica não modal. Fazendo com que o 
problema seja resolvido sem precisarmos atentar a aspectos temporais. Porém, existem sistemas aonde
o aspecto temporal se faz presente. Sistemas reativos são um exemplo disto.

A seguir será elucidado um dos problemas clássicos aonde o tempo se faz presente.
\subsection{Problema Escolhido}
O propósito do Sistema de Aquecimento de Água (SAA)[citação] é o de assegurar o funcionamento 
seguro do Aquecedor de Água. O aquecedor de água opera em segurança se o nível da água
não exceder o limíte de tolerância. Além desta restrição, o tanque do sistema tem também 
sua resistência a pressão, que não pode ser maior que um valor estipulado pelo fabricante.

O SAA é composto por uma série de sistemas que são necessários para o aquecimento da água e também
para o monitoramento das condições de operação. A seguir são listados os principais subsistemas:
\begin{itemize}
\item um tanque para o aquecimento da água
\item dispositivo para a medição do nível da água
\item uma bomba d'água para o abastecimento do tanque
\item um dispositivo para a medição da bomba d'água
\item um dispositivo para medir a pressão dentro do tanque
\item um sistema de controle para atuar nos dispositivos.
\end{itemize}

\subsection{Modelagem Formal}
intro dizendo que é uma POC, faremos a modelagem para ver os pontos fortes
O SAA é composto de diversos subsistemas. Abordaremos apenas o de aquecimento
\subsubsection{Modelo CSP}
\subsubsection{Modelo Alloy}
\subsubsection{Modelo State Charts}
