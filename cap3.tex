\section{Proposta}
\subsection{Proposta TCC 1}
\subsubsection{Objetivos}
Como objetivos da primeira etapa do trabalho de conclusão, buscamos, em primeiro lugar,
elevar nosso conhecimento nas áreas de sistemas de tempo real e especialmente
em métodos formais, com suas ferramentas e aplicações. Iremos também ,
baseado no conhecimento de pesquisa feita no decorrer deste trabalho, modelar um sistema
nos diferentes métodos formais de modelagem descritos na seção [CITAR SEÇÃO] para fins de
prova de conceito, possibilitando a avaliação destes com base nessas experiências de implementação,
analisando quais são os seus pontos fortes e fracos e quais ferramentas melhor se adaptam às nossas
necessidades para implementarmos a segunda etapa desse trabalho de conclusão.
\\

\paragraph{Sistema Escolhido}\mbox{} \\\\
No mundo da computação vários problemas são resolvidos sem levar em conta o tempo.
Isto decorre da natureza dos problemas obedecerem a lógica não modal, fazendo com que o
problema seja resolvido sem precisarmos atentar a aspectos temporais. Apesar disso,
existem sistemas aonde o aspecto temporal se faz presente, sistemas reativos são um exemplo disto.

Para a análise preliminar dos métodos formais de modelagem propostos, foi escolhido um problema que
já é conhecido pela literatura, exposto por Abrial, B\"{o}rger e Langmaack~\cite{opac-b1092561},
para que seja possível utilizar como exemplo de problema a ser modelado.

O propósito do \textit{Sistema de Aquecimento de Água} (SAA) é o de garantir o funcionamento,
de forma segura, do aquecedor de água. O aquecedor de água opera em segurança se o nível da água
não exceder o limite de tolerância. Além desta restrição, o tanque do sistema tem também
sua resistência à pressão, que não pode ser maior que um valor estipulado pelo fabricante.

O SAA é composto por uma série de sistemas que são necessários para o aquecimento da água e também
para o monitoramento das condições de operação. A seguir são listados os principais subsistemas:
\begin{itemize}
\item Um tanque para o aquecimento da água.
\item Dispositivo para a medição do nível da água.
\item Uma bomba d'água para o abastecimento do tanque.
\item Um dispositivo para a medição da bomba d'água.
\item Um dispositivo para medir a pressão dentro do tanque.
\item Um sistema de controle para atuar nos dispositivos.
\end{itemize}

Para criação da prova de conceito em relação aos métodos formais de modelagem de sistemas préviamente
mencionada, utilizaremos um destes subsistemas, nos capacitando tanto para a escolha do método quanto
para o conhecimento das linguagens, ferramentas e técnicas utilizadas para essa finalidade de modelagem.

\subsubsection{Cronograma}
\renewcommand{\arraystretch}{1.5}
\definecolor{lightgray}{gray}{0.9}
\rowcolors{2}{white}{lightgray}

\newcolumntype{C}{>{\centering\arraybackslash}X}

\begin{tabularx}{\textwidth}{ | c | C | }
\hline
\textbf{Data} & \multicolumn{1}{c|}{\textbf{Evento}} \\
\hline
xx.xx.2014 & Modelagem formal do sistema utilizando Statecharts \\
xx.xx.2014 & Modelagem formal do sistema utilizando Alloy \\
xx.xx.2014 & Modelagem formal do sistema utilizando CSP \\
24.11.2014 & Entrega do Volume Final \\
\hline
\end{tabularx}


\subsubsection{Lista de Atividades}


\subsection{Proposta TCC 2}
\subsubsection{Objetivos}
\subsubsection{Cronograma}
\subsubsection{Lista de Atividades}
