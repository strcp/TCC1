\section{Proposta}
\subsection{Proposta TCC 1}
\subsubsection{Objetivos}

Educação em MF.
Justificar o MF que será utilizado no TCC2.
(Manteremos o texto atual com modificações para explicar o que iremos modelar)

No mundo da computação vários problemas são resolvidos sem levar em conta o tempo.
Isto decorre da natureza dos problemas obedecerem a lógica não modal, fazendo com que o
problema seja resolvido sem precisarmos atentar a aspectos temporais. Apesar disso,
existem sistemas aonde o aspecto temporal se faz presente, sistemas reativos são um exemplo disto.

A seguir será elucidado um dos problemas clássicos aonde o tempo se faz presente.
\paragraph{Problema Escolhido}
Para uma análise preliminar, foi escolhido um problema que já é conhecido pela literatura, exposto
por Abrial, B\"{o}rger e Langmaack~\cite{opac-b1092561}, para que seja possível utilizar como exemplo
de problema a ser modelado.

O propósito do \textit{Sistema de Aquecimento de Água} (SAA) é o de garantir o funcionamento,
de forma segura, do aquecedor de água. O aquecedor de água opera em segurança se o nível da água
não exceder o limite de tolerância. Além desta restrição, o tanque do sistema tem também
sua resistência à pressão, que não pode ser maior que um valor estipulado pelo fabricante.

O SAA é composto por uma série de sistemas que são necessários para o aquecimento da água e também
para o monitoramento das condições de operação. A seguir são listados os principais subsistemas:
\begin{itemize}
\item Um tanque para o aquecimento da água.
\item Dispositivo para a medição do nível da água.
\item Uma bomba d'água para o abastecimento do tanque.
\item Um dispositivo para a medição da bomba d'água.
\item Um dispositivo para medir a pressão dentro do tanque.
\item Um sistema de controle para atuar nos dispositivos.
\end{itemize}

Neste trabalho será feita a modelagem de um dos subsistemas do sistema descrito acima.
Para fins de prova de conceito, serão utilizados três métodos de modelagem para que seja possível avaliar,
com base nessas experiências de implementação, quais são os pontos fortes e fracos de cada uma e qual ferramenta
utilizar em uma modelagem maior em trabalho futuro.
\paragraph{Modelo CSP}
\paragraph{Modelo Alloy}
\paragraph{Modelo State Charts}

\subsubsection{Cronograma}
\subsubsection{Lista de Atividades}


\subsection{Proposta TCC 2}
\subsubsection{Objetivos}
\subsubsection{Cronograma}
\subsubsection{Lista de Atividades}
