\section{Hellfire and COMPaaS Integration Layer (HAC)}
\label{sec:HAC}
Atualmente, a plataforma COMPaaS possui a necessidade de criação de uma abstração para
cada dispositivo que pretende se comunicar à ela, tornando o processo de adoção do COMPaaS
oneroso neste aspecto. Para eliminar esta necessidade, propõe-se a adaptação do sistema
HellfireOS, que já possui a infraestrutura necessária para suportar múltiplos tipos de
sensores, para que suporte a comunicação com a plataforma COMPaaS de forma praticamente
transparente do ponto de vista do suporte de hardware dos sensores ligados ao sistema,
permitindo que os produtos que adotem o HellfireOS como seu sistema embarcado estejam aptos
a se integrarem à solução IoT de forma mais rápida e simples.

Para possibilitar essa comunicação, será implementada uma camada de integração entre o sistema de tempo real
, previamente descrito na seção~\ref{sec:HellfireOS}, HellfireOS e a plataforma COMPaaS, também analisada
anteriormente na seção~\ref{sec:COMPaaS}. Essa camada consistirá em adicionar suporte a duas questões
críticas no HellfireOS, uma biblioteca compatível com o protocolo de comunicação esperado pela camada de
mais baixo nível do COMPaaS e, para prova de conceito, uma meio de comunicação entre o HellfireOS e a
plataforma IoT, tendo em vista que o sistema operacional ainda não possui uma implementação de todas as
camadas de rede necessárias, como por exemplo, o protocolo TCP.

\begin{figure}[H]
	\centering
		\includegraphics[width=0.3\textwidth]{fig/COMPaaS_HF.png}
	\caption{Arquitetura em alto nível da integração com a camada de integração.}
\end{figure}

Para fins de validação da camada de integração, será criado um protótipo utilizando uma placa
CubieBoard rodando um sistema Linux que rodará o sistema HellfireOS virtualizado. O meio de comunicação
entre o host Linux e o HellfireOS será feito através de um espaço de memória compartilhada, que será
abstraída da camada de integração desenvolvida, permitindo que a mesma camada seja utilizada com
poucas ou nenhuma alteração quando o HellfireOS possuir outros meios de comunicação via rede.
Para possibilitar que essa comunicação entre host e HellfireOS funcione, será desenvolvida uma
daemon no sistema host para interfacear sua camada de comunicação via rede com a comunicação
via memória compartilhada com o sistema de tempo real.

\begin{figure}[H]
	\centering
		\includegraphics[width=0.6\textwidth]{fig/HAC_IL.png}
	\caption{Arquitetura do protótipo proposto para prova de conceito.}
	\label{fig:HAC_IL}
\end{figure}

\subsection{Prova de Conceito}
Para provar o funcionamento da camada de integração, será utilizada uma
simulação de sensor médico baseado em funções matemáticas suportadas pelo HellfireOS
ou mesmo um sensor real, caso haja o suporte Cubieboard, HellfireOS e sensor disponível.
