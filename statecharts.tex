Statecharts~\cite{Harel:1987:SVF:34884.34886}, é um formalismo criado por
David Harel nos anos 80 que possibilita descrever, de forma rigorosa como a
álgebra de processos, sistemas complexos de forma gráfica, possibilitando uma
visualização melhor do sistema em questão.

A grande parte da literatura que compreende o desenvolvimento de sistemas atende
apenas às necessidades de sistemas transformacionais, State charts foi criado para
atacar este déficit, atendendo aos interesses de sistemas reativos orientado a eventos.

\subsubsection{Ferramentas}
\begin{itemize}

\item{YAKINDU Statechart Tools (SCT)}

É uma ferramenta open source que provê um ambiente de modelagem integrado para especificação e
desenvolvimento de sistemas baseados no conceito do Statecharts. Capaz de modelar os estados, transições
e hierarquia de estados de maneira gráfica, porém, todas as declarações e ações utilizam uma notação
textual. Efetua sua validação de restrições durante a edição, provendo informação imediata ao usuário.


\item{STATEMATE}

\end{itemize}
