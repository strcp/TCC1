Statecharts~\cite{Harel:1987:SVF:34884.34886}, é um formalismo criado por
David Harel nos anos 80 que possibilita descrever, de forma rigorosa como a
álgebra de processos, sistemas complexos de forma gráfica, possibilitando uma
visualização melhor do sistema em questão.

A grande parte da literatura que compreende o desenvolvimento de sistemas atende
apenas às necessidades de sistemas transformacionais, State charts foi criado para
atacar este déficit, atendendo aos interesses de sistemas reativos orientado a eventos.

\paragraph{Ferramentas}
\begin{itemize}

\item{YAKINDU Statechart Tools (SCT)}

É uma ferramenta open source que provê um ambiente de modelagem integrado para especificação e
desenvolvimento de sistemas baseados no conceito do Statecharts. Capaz de modelar os estados, transições
e hierarquia de estados de maneira gráfica, porém, todas as declarações e ações utilizam uma notação
textual. Efetua sua validação de restrições durante a edição, provendo informação imediata ao usuário.


\item{STATEMATE}

Uma ferramenta elaborada pela empresa \textit{I-Logix}, que pertencia à David Harel, o próprio
idealizador do formalismo \textit{Statecharts}. Com ela o usuário tem a possibilidade de desenhar
e analisar o modelo do sistema. Além disso, a ferramenta permite gerar automaticamente
código executável baseado nesse modelo, em Ada e em C.

Em 2007, o time que desenvolveu o Statemate ganhou o \textit{ACM Software System Award}. A afirmação
da ACM no anúncio do prêmio foi que o Statemate foi o primeiro software que conseguiu superar os
desafios de sistemas de tempo real complexos e interativos, conhecidos como sistemas reativos.

Atualmente o Statemate é parte da unidade da IBM chamada \textit{Rational Software}.

\end{itemize}
