Segundo Steve Schneider~\cite{Schneider:1999:CRT:555233}, a linguagen de \textit{Communicating Sequential Processes}
(CSP) foi desenvolvida para possibilitar a descrição de padrões de interação em sistemas concorrentes,
possuindo teorias subjacentes para raciocinar e avaliar os componentes desses sistemas.
\paragraph{}
A linguagem foi descrita primeiramente em um artigo de Tony Hoare~\cite{Hoare:1978:CSP:359576.359585} em 1978,
basicamente apenas como uma linguagem de programação e não tanto como uma álgebra de processos, o que veio a ocorrer
apenas anos mais tarde com o desenvolvimento e refinamentos propostos por Stephen Brookes e Andrew William Roscoe~\cite{Brookes:1984:TCS:828.833},
culminando na sua forma de álgebra de processos utilizada atualmente.
Desde sua criação, o CSP vem evoluindo de forma constante e inclusive influenciando no design de algumas linguagens
de programação como Limbo e Go, elaborada pela empresa Google.
\paragraph{}
O CSP tem sido aplicado de maneira prática na indústria como ferramenta de verificação e especificação dos aspectos
de concorrência nos mais diferentes tipos de sistemas e áreas, incluindo sistemas de segurança para ecommerce, como
publicado no artigo de Anthony Hall e Roderick Chapman~\cite{976937}.

\subsubsection{Ferramentas}
\paragraph{}
http://www.cs.ox.ac.uk/ucs/CSPtools.html
\begin{itemize}
\item{CSP Prover}
\item{FDR (Failures Divergences Refinement)}
\item{ProBE (Process Behaviour Explorer)}
\item{ARC (Adelaide Refinement Checker)}
\end{itemize}
