\section{Sistemas de Tempo Real}
\label{sec:STR}

\subsection{Definição de um Sistema de Tempo Real}
\label{sec:DefSTR}
Um sistema de tempo real, ao contrário do que se costuma pensar, não tem
como objetivo uma execução necessariamente rápida, mas sim, previsível.
Deste tipo de sistema, podemos encontrar duas características fundamentais:

\begin{itemize}
\item Um sistema de tempo real precisa produzir resultados computacionais corretos
(chamados de corretude logica ou funcional).
\item Os resultados computacionalmente corretos obtidos precisam ser concluídos em
um período de tempo pré-definido, caracterizando a previsibilidade na execução da tarefa.
\end{itemize}

Sistemas de tempo real são definidos como sistemas nos quais a corretude nos
resultados de execução de maneira geral são dependentes tanto da corretude
lógica como da previsibilidade do tempo de execução, sendo assim, a previsibilidade
do tempo de execução é, ao menos, tão importante quanto a corretude funcional
nesse tipo de sistemas.~\cite{Li:2003:RCE:829584}

Os sistemas de tempo real são utilizados para atender à tarefas que possuem algum tipo
de restrição temporal em sua execução. Esse tipo de necessidade está muito presente no
dia-a-dia, e são aplicados aos mais diversos tipos de tarefas, desde controladores de
leitores de DVDs, elevadores, freios de carro, controladores de mísseis e até piloto automático
de aeronaves.

\subsection{Tipos de Sistema de Tempo Real}
Como abordado na seção~\ref{sec:DefSTR}, para que um sistemas de tempo real obtenha uma
execução de tarefa correta, é necessário terminar essa execução em um período pré-definido,
chamado de \textit{deadline}. Portanto, é possível afirmar que este tipo de sistema, por possuir
essa restrição temporal, é regido pelos \textit{deadlines} de suas tarefas.

Devido a importância dos \textit{deadlines}, esse tipo de sistema pode ser classificado como
críticos ou não críticos, baseado na tolerância de \textit{deadlines} perdidos, a utilidade
dos resultados computados após a perda do \textit{deadline} e a severidade da penalidade em
perder um prazo de execução.

\subsubsection{Sistemas de Tempo Real Críticos}
Os sistemas de tempo real chamados de críticos são aqueles que possuem uma restrição severa
de prazo na execução de tarefas, ou seja, sua tolerância em perder prazos é muito pequena
ou nenhuma. Em muitos desses sistemas, as informações computadas fora do prazo são consideradas
inúteis, caracterizando uma penalidade grave em perder o prazo de execução.

Um sistema de tempo real crítico é um sistema que precisa executar suas tarefas dentro do prazo
pré-definido com uma tolerância muito próxima de zero. Os prazos devem ser atendidos, ou os resultados
são catastróficos. O custo da perda de prazos na execução possuem um custo muito alto, podendo inclusive
envolver vidas humanas. Os resultados computados após o prazo pré-definido possuem uma utilidade próxima
de zero ou possuem um grande grau de depreciação com a decorrência do tempo após o prazo.~\cite{Li:2003:RCE:829584} \\\\
Alguns exemplos de sistemas de tempo real considerados críticos são:
\begin{itemize}
\item Sistema de navegação de aeronaves.
\item Freios de carro (ABS).
\item Marca-passo.
\item Controladores de mísseis.
\end{itemize}

\subsubsection{Sistemas de Tempo Real Não Críticos}
Os sistemas de tempo real considerados não críticos são aqueles aonde uma perda de prazo
resulta em uma penalidade leve, como uma distorção em uma música sendo lida de um CD, perda
de alguns frames em um vídeo entre outras consequências de baixa criticidade.

Esses sistemas precisam atender seus prazos pré-definidos, porém com um certo grau de flexibilidade.
Os prazos podem possuir diferentes graus de tolerância, prazos balizados com tempo médio e até em avaliação
estatística dos tempos de resposta. Nesse tipo de sistema, apesar da perda de prazos não resultar em uma
falha no sistema, os custos da perda dos mesmos pode se tornar grande, dependendo da proporção em que isso
ocorre.~\cite{Li:2003:RCE:829584} \\\\
Alguns exemplos de sistemas de tempo real considerados não críticos são:
\begin{itemize}
\item Sistema responsável por leitura de DVD.
\item Sistema de som em computadores pessoais.
\item Decodificador de sinal de televisão.
\end{itemize}

\subsection{Validação e Verificação}
Esse tipo de tecnologia precisa possuir a garantia de que o sistema desenvolvido está apto a ser utilizado em produção
e que conseguirá atender à todas as demandas que o serão requisitadas em seu ambiente final de uso.

Segundo Kopetz~\cite{Kopetz:1997:RSD:523911}, até 50\% dos custos de desenvolvimento de um sistema de tempo real, é aplicado
para garantir que o sistema atende completamente à seu propósito. Em aplicações de alta criticidade que necessitam ser certificadas,
essa porcentagem se torna ainda maior.

Para que seja possível garantir essas restrições, dois conceitos são amplamente aplicados,
\textit{verificação}  e \textit{validação}.

A validação atua na avaliação da consistência entre o modelo informal das intenções do usuário e
o comportamento do sistema sendo testado, enquanto a verificação consiste em analisar a consistência
entre a especificação formal criada e o próprio sistema em teste. Enquanto a verificação pode ser
reduzida a um processo formal, a validação precisa avaliar o comportamento do sistema em relação ao mundo real
aonde é aplicado. O principal método de validação é o teste, enquanto o principal método de verificação é a
análise formal.

\subsubsection{Validação}
A validação do sistema busca garantir que o produto desenvolvido atende às necessidades dos usuários. Um dos processos
mais utilizados para esse tipo de análise chama-se \textit{teste de aceitação do usuário}.

O teste de aceitação do usuário é um dos estágios finais do do projeto. Consiste em colocar o sistema a funcionar em seu ambiente final,
sendo operado por seu usuário final verdadeiro, avaliando suas ações e reações durante a utilização. Esse estágio de teste
não procura encontrar erros cosméticos ou mesmo de quebra de software já que esse tipo de problema deve ser eliminado em
estágios anteriores da qualificação do software.

\subsubsection{Testes Padrão de Sistemas}
Segundo Roger S. Pressman~\cite{pre_2005}, o design de caso de testes para sistemas de tempo real pode ser dividido
em quatro passos:

\begin{enumerate}
\item \textbf{Teste de Tarefas}

No princípio, cada tarefa é testada individualmente com maneiras convencionais de testes estáticos, buscando encontrar
apenas erros de lógica ou sintaxe no programa. É um tipo de teste que não leva em conta, especificamente, as peculiaridades
dos sistemas de tempo real, não considerando, ainda, as restrições temporais impostas ao sistema e suas tarefas.


\item \textbf{Teste de Comportamento}

Utilizando os modelos de sistema desenvolvidos com ferramentas para automação de testes, é possível simular os comportamentos
de um sistema de tempo real e os impactos de eventos externos sobre esses comportamentos.


\item \textbf{Teste Intertarefa}

Partindo do pressuposto de que as tarefas estão livres de erros, sendo isso validado pelo \textit{teste de tarefas}, as mesmas são
testadas adicionando ao cenário as características de restrição temporal impostas ao sistema. Esse tipo de teste visa encontrar
erros de comunicação interprocesso, testando tarefas assíncronas com diferentes taxas e tamanhos de dados.


\item \textbf{Teste do Sistema}

Nesse ponto, software e hardware são integrados e uma larga escala de testes de sistema são feitos para encontrar erros durante
a comunicação entre software e hardware.

\end{enumerate}

\subsubsection{Verificação Formal}
A verificação formal consiste em, utilizando uma base matemática de métodos formais, apresentar uma prova ou
contra-prova da corretude de algoritmos de um sistema descrito em uma dada \textit{especificação formal}, melhor
descrita no capítulo~\ref{sec:MFeSTR}.

Este assunto será mais aprofundado, descrevendo suas técnicas em detalhes na segunda etapa deste trabalho.

