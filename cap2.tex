\section{Internet of Things}
\label{sec:IoT}
Citação~\cite{COMPaaS}

\subsection{Aplicações}
\label{sec:IoTAp}
\subsubsection{Gerência de Energia}
\subsubsection{Transportes}
\subsubsection{Automação Residencial}
\subsubsection{Sistemas médicos e de cuidados pessoais de saúde}

\subsection{Plataformas Existentes}
\label{sec:IoTPlataformas}

\subsubsection{mbed}
A empresa ARM provê uma solução completa de \textit{Internet of Things} (IoT) chamada mbed\cite{mbed}. Esta plataforma é dividida em três módulos, mbed OS, mbed Device Server e mbed Tools.\\
A plataforma utiliza em seu sistema embarcado a tecnologia mbedTLS para criptografia de dados com baixo consumo de memória, desenvolvida inicialmente sob o nome de PolarSSL pela empresa holandesa Offspark, que foi comprada recentemente pela ARM.\\
A seguir detalharemos cada uma das tecnologias oferecidas pela mbed.\\

\subsubsubsection{mbed OS}
É um sistema operacional, disponiblizado de graça, para processadores da linha ARM Cortex-M, que são desenvolvidos visando a eficiência de energia e produtividade.\\
A arquitetura do mbed OS fornece componentes de sofrtware e um framework de aplicação que, combinados com a grande diversidade de empresas e desenvolvedores que disponiblizam bibliotecas e drivers, facilitam e agilizam o processo de desenvolvimento de aplicações para IoT.\\
O mbed OS provê suporte aos padrões como Bluetooth Smart®, Cellular, Thread, Wi-Fi®, e 802.15.4/6LoWPAN junto com TLS/DTLS, CoAP, HTTP, MQTT e Lightweight M2M

\subsubsubsection{mbed Device Server}
É um produto que precisa de licença, serve para conectar e gerenciar os dispositivos de uma forma segura. Ele provê a ligação entre os os protocolos utilizados nos dispositivos IoT e a API que é utilizada por desenvolvedores web. Isto simplifica a integração de dispositivos IoT que geram \lq little data\rq\ que vão para os servidores que analizam e agregam a informação gerando a \lq big data\rq.\\
O Device Server é escalável, podendo conectar e gerenciar milhões de dispositivos.

\subsubsubsection{Arquitetura}
https://mbed.org/technology/device-server/ \\

\subsubsection{FlowCloud}
\subsubsection{Arrayent}
www.arrayent.com\\
http://en.wikipedia.org/wiki/Arrayent\\

\subsubsection{Samsung IoT Plataform}
http://developer.samsung.com/iot \\

\subsubsection{IBM Internet of Things Foundation}
http://www-03.ibm.com/software/products/en/internet-of-things-foundation \\

\subsubsection{Carriots}
www.carriots.com\\
