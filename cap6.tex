\section{Proposta}
\subsection{Proposta TCC 1}
\subsubsection{Objetivos}
\label{sec:objetivos_1}
Esta primeira etapa do trabalho de conclusão possui como objetivos, em primeiro lugar,
elevar o conhecimento nas áreas de sistemas embarcados e \textit{Internet of Things}.
Para tal, será feita uma ampla pesquisa do estado da arte nas áreas em questão,
aprofundando o conhecimento técnico nos respectivos assuntos.
O objetivo secundário desta etapa visa obter todas as informações e equipamentos necessários
para a criação dos ambiente que serão utilizados na segunda etapa do projeto.

\subsubsection{Cronograma}
\renewcommand{\arraystretch}{1.5}
\definecolor{lightgray}{gray}{0.9}
\rowcolors{2}{white}{lightgray}

\newcolumntype{C}{>{\centering\arraybackslash}X}

\begin{tabularx}{\textwidth}{ | c | C | }
\hline
\textbf{Data} & \multicolumn{1}{c|}{\textbf{Evento}} \\
\hline
05.05.2015 & Aprofundamento no estado da arte de \textit{Internet of Things}. \\
20.05.2015 & Aprofundamento nos componentes utilizados para a segunda etapa do projeto. \\
22.06.2015 & Entrega do Volume Final de TC1. \\
\hline
\end{tabularx}

\subsubsection{Lista de Atividades}
\begin{itemize}
\item Buscar o hardware necessário para as implementações.
\item Busca de referencial teórico sobre o estado da arte em \textit{Internet of Things}.
\item Busca por documentação técnica das diferentes soluções pesquisadas.
\end{itemize}

\subsection{Proposta TCC 2}
\subsubsection{Objetivos}
\label{sec:objetivos_2}
Ao final do projeto, objetiva-se permitir que o HellfireOS se integre à plaforma COMPaaS,
permitindo que os produtos que possuam o HellfireOS como seu sistema embarcado tenham
uma integração mais rápida e simples de seus dispositivos sensores à plataforma IoT.

Nesta etapa final, serão finalizados criação e ajustes necessários do ambiente proposto para teste
para então implementar a integração da plataforma COMPaaS com o sistema HellfireOS,
descritos nas seções~\ref{sec:COMPaaS} e~\ref{sec:HellfireOS} respectivamente, com a criação
do \textit{Hellfire and COMPaaS Integration Layer} (HAC), descrito na seção~\ref{sec:HAC},
objetivando a criação de uma solução completa, desde o sistema operacional embarcado
responsável pela gerência de múltiplos sensores, até a plataforma de IoT, possibilitando
um uso de funcionalidade robusta e simples do HellfireOS na criação e aplicação de
equipamentos integráveis no conceito de \textit{Internet of Things}.
Para prova e testes desta integração, será criado um simulador de sensor para área médica,
baseado em estudos efetuados nesta segunda etapa do trabalho, possibilitando um teste
da integração de forma completa, passando por todos os níveis descritos nas arquiteturas
tanto da plataforma COMPaaS como no HellfireOS nas seções anteriores.

\subsubsection{Cronograma}
\renewcommand{\arraystretch}{1.5}
\definecolor{lightgray}{gray}{0.9}
\rowcolors{2}{white}{lightgray}

\newcolumntype{C}{>{\centering\arraybackslash}X}

\begin{tabularx}{\textwidth}{ | c | C | }
\hline
\textbf{Data} & \multicolumn{1}{c|}{\textbf{Evento}} \\
\hline
07.04.2015 & Criação do ambiente da Cubieboard com o host Linux. \\
13.08.2015 & Implementação da comunicação entre host e HellfireOS através de memória compartilhada. \\
21.08.2015 & Criação da \textit{daemon} para interfacear a memória compartilhada do host e a rede.  \\
09.10.2015 & Término do desenvolvimento do HAC Integration Layer no HellfireOS. \\
06.11.2015 & Criação do simulador de sensor para prova de conceito da integração. \\
13.11.2015 & Teste completo da implementação. \\
22.11.2015 & Entrega do Volume Final. \\
09.12.2015 & Apresentação do Trabalho. \\
\hline
\end{tabularx}

\subsubsection{Lista de Atividades}
\begin{itemize}
\item Escolha da distribuição adequada para rodar no hardware.
\item Testes para prova da implementação da comunicação entre host e HellfireOS.
\item Definição do funcionamento da \textit{daemon} proposta.
\item Testes da implementação da \textit{daemon} de comunicação entre memória compartilhada e rede.
\item Testes do funcionamento da implementação do HAC Integration Layer.
\item Definição das funcionalidades do sensor a ser simulado.
\item Teste para prova da implementação da integração.
\end{itemize}

