\section{Proposta}
\subsection{Proposta TCC 1}
\subsubsection{Objetivos}
\label{sec:objetivos_1}
Esta primeira etapa do trabalho de conclusão possui como objetivos, em primeiro lugar,
elevar o conhecimento nas áreas de sistemas embarcados. Baseado no conhecimento de pesquisa
feita no decorrer deste trabalho, será criado o ambiente necessário para efetuar a prova de conceito,
obtendo o hardware necessário, e fazendo as instalações necessárias de sistemas operacionais, host
Linux e o HellfireOS, este de forma virtualizada. Também serão iniciadas as medidas necessárias
para suprir a falta de das camadas de protocolos de rede necessárias, implementando uma comunicação
através de memória compartilhada entre os dois sistemas e, inclusive, uma deamon para interfacear
o espaço de memória compartilhada no host Linux e a rede.

\subsubsection{Cronograma}
\renewcommand{\arraystretch}{1.5}
\definecolor{lightgray}{gray}{0.9}
\rowcolors{2}{white}{lightgray}

\newcolumntype{C}{>{\centering\arraybackslash}X}

\begin{tabularx}{\textwidth}{ | c | C | }
\hline
\textbf{Data} & \multicolumn{1}{c|}{\textbf{Evento}} \\
\hline
22.04.2015 & Criação do ambiente da Cubieboard com o host Linux. \\
04.04.2015 & Preparação do ambiente com o suporte e configurações de virtualização necessárias. \\
20.05.2015 & Instalação do HellfireOS de forma virtualizada rodando sobre o host Linux. \\
21.05.2015 & Início da implementação da comunicação entre host e HellfireOS através de memória compartilhada. \\
22.06.2015 & Entrega do Volume Final de TC1. \\
\hline
\end{tabularx}

\subsubsection{Lista de Atividades}
\begin{itemize}
\item Conseguir o hardware necessário para as implementações.
\item Escolha da distribuição adequada para rodar no hardware.
\item Instalação do host no hardware.
\item Pesquisa e configuração do suporte de virtualização no host.
\item Pesquisa e instalação do HellfireOS como sistema virtualizado.
\item Implementação da comunicação entre host e HellfireOS.
\end{itemize}

\subsection{Proposta TCC 2}
\subsubsection{Objetivos}
\label{sec:objetivos_2}
Ao final do projeto, objetiva-se permitir que o HellfireOS se integre à plaforma COMPaaS,
permitindo que os produtos que possuam o HellfireOS como seu sistema embarcado tenham
uma integração mais rápida e simples de seus dispositivos sensores à plataforma IoT.

Nesta etapa final, serão finalizados os ajustes necessários do ambiente proposto para teste
para então implementar a integração da plataforma COMPaaS com o sistema HellfireOS,
descritos nas seções~\ref{sec:COMPaaS} e~\ref{sec:HellfireOS} respectivamente, com a criação
do \textit{Hellfire and COMPaaS Integration Layer} (HAC), descrito na seção~\ref{sec:HAC},
objetivando a criação de uma solução completa, desde o sistema operacional embarcado
responsável pela gerência de múltiplos sensores, até a plataforma de IoT, possibilitando
um uso de funcionalidade robusta e simples do HellfireOS na criação e aplicação de
equipamentos integráveis no conceito de \textit{Internet of Things}.
Para prova e testes desta integração, será criado um simulador de sensor da área médica,
baseada em estudos efetuados nesta segunda etapa do trabalho, possibilitando um teste
da integração de forma completa, passando por todos os níveis descritos nas arquiteturas
tanto da plataforma COMPaaS como no HellfireOS nas seções anteriores.

\subsubsection{Cronograma}
\renewcommand{\arraystretch}{1.5}
\definecolor{lightgray}{gray}{0.9}
\rowcolors{2}{white}{lightgray}

\newcolumntype{C}{>{\centering\arraybackslash}X}

\begin{tabularx}{\textwidth}{ | c | C | }
\hline
\textbf{Data} & \multicolumn{1}{c|}{\textbf{Evento}} \\
\hline
07.08.2015 & Implementação da comunicação entre host e HellfireOS através de memória compartilhada. \\
21.08.2015 & Criação da daemon para interfacear a memória compartilhada do host e a rede.  \\
09.10.2015 & Término do desenvolvimento do HAC Integration Layer no HellfireOS. \\
06.11.2015 & Criação do simulador de sensor para prova de conceito da integração. \\
13.11.2015 & Teste completo da implementação. \\
22.11.2015 & Entrega do Volume Final. \\
09.12.2015 & Apresentação do Trabalho. \\
\hline
\end{tabularx}

\subsubsection{Lista de Atividades}
\begin{itemize}
\item Testes para prova da implementação da comunicação entre host e HellfireOS.
\item Definição do funcionamento da daemon proposta.
\item Testes da implementação da daemon de comunicação entre memória compartilhada e rede.
\item Testes do funcionamento da implementação do HAC Integration Layer.
\item Definição das funcionalidades do sensor a ser simulado.
\item Teste para prova da implementação da integração.
\end{itemize}

