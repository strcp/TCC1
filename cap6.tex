\section{Proposta}
\subsection{Proposta TCC 1}
\subsubsection{Objetivos}
Como objetivos da primeira etapa do trabalho de conclusão, buscamos, em primeiro lugar,
elevar nosso conhecimento nas áreas de sistemas embarcados. Iremos também,
baseado no conhecimento de pesquisa feita no decorrer deste trabalho, criar o ambiente
necessário para efetuar a prova de conceito, obtendo o hardware necessário, e fazendo
as instalações necessárias de sistemas operacionais, host Linux e o HellfireOS, este de forma
virtualizada. Iremos iniciar também as medidas necessárias para suprir a falta de da completude
das camadas de protocolos de rede necessária, implementando uma comunicação através de memória
compartilhada entre os dois sistemas e, inclusive, uma deamon para interfacear o espaço de memória
compartilhada no host Linux e a rede.

\subsubsection{Cronograma}
\renewcommand{\arraystretch}{1.5}
\definecolor{lightgray}{gray}{0.9}
\rowcolors{2}{white}{lightgray}

\newcolumntype{C}{>{\centering\arraybackslash}X}

\begin{tabularx}{\textwidth}{ | c | C | }
\hline
\textbf{Data} & \multicolumn{1}{c|}{\textbf{Evento}} \\
\hline
16.04.2015 & Criação do ambiente da Cubieboard com o host Linux. \\
17.04.2015 & Preparação do ambiente com o suporte e configurações de virtualização necessárias. \\
17.04.2015 & Instalação do HellfireOS de forma virtualizada rodando sobre o host Linux. \\
17.04.2015 & Início da implementação da comunicação entre host e HellfireOS através de memória compartilhada. \\
22.06.2015 & Entrega do Volume Final. \\
\hline
\end{tabularx}

\subsubsection{Lista de Atividades}
\begin{itemize}
\item Escolher....
\end{itemize}


\subsection{Proposta TCC 2}
\subsubsection{Objetivos}
Nesta etapa iremos finalizar os ajustes necessários do ambiente proposto para teste para então
implementar a integração da plataforma COMPaaS com o sistema HellfireOS,
descritos nas seções~\ref{sec:COMPaaS} e~\ref{sec:HellfireOS} respectivamente, com a criação
do \textit{Hellfire and COMPaaS Integration Layer} (HAC), descrito na seção~\ref{sec:HAC},
objetivando a criação de uma solução completa, desde o sistema operacional embarcado
responsável pela gerência de múltiplos sensores, até a plataforma de IoT, possibilitando
um uso de funcionalidade robusta e simples do HellfireOS na criação e aplicação de
equipamentos integráveis no conceito de \textit{Internet of Things}.
Para prova e testes desta integração, será criado um simulador de sensor da área médica,
baseada em estudos efetuados nesta segunda etapa do trabalho, possibilitando um teste
da integração de forma completa, passando por todos os níveis descritos nas arquiteturas
tanto da plataforma COMPaaS como no HellfireOS nas seções anteriores.

\subsubsection{Cronograma}
\renewcommand{\arraystretch}{1.5}
\definecolor{lightgray}{gray}{0.9}
\rowcolors{2}{white}{lightgray}

\newcolumntype{C}{>{\centering\arraybackslash}X}

\begin{tabularx}{\textwidth}{ | c | C | }
\hline
\textbf{Data} & \multicolumn{1}{c|}{\textbf{Evento}} \\
\hline
xx.08.2015 & Escolha de.... \\
xx.08.2015 & Escolha de.... \\
xx.08.2015 & Escolha de.... \\
xx.08.2015 & Escolha de.... \\
xx.11.2015 & Entrega do Volume Final \\
xx.11.2015 & Apresentação do Trabalho \\
\hline
\end{tabularx}

\subsubsection{Lista de Atividades}
