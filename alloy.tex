Alloy é uma linguagem formal de especificação utilizada para expressar sistemas que possuem
estruturas e comportamentos com restrições complexas. As fundações matemáticas deste formalismo
foram muito influenciadas pela notação Z~\cite{opac-b1091336}, porém sua sintaxe foi influenciada
por linguagens como \textit{Object Constraint Language}(OCL), buscando ser mais simples para
criação de modelos de forma incremental.

A primeira versão da linguagem Alloy surgiu em 1997 e era apenas uma limitada linguagem para modelagem
de objetos. Sua evolução aconteceu mais tarde com a adição de quantificadores, polimorfismo, subtipos e
assinaturas. Essa linguagem se diferencia de boa parte das linguagens formais de especificação desenvolvidas
para verificação de modelos no sentido de que aceita a definição de modelos infinitos.

\subsubsection{Ferramentas}
\begin{itemize}
\item{Alloy Analyzer}

É uma ferramenta que analisa especificações escritas na linguagem Alloy. Como a linguagem, essa ferramenta
foi desenvolvida no \textit{Massachusetts Institute of Technology} (MIT). Uma das grandes vantagens do Alloy
Analyzer é sua capacidade de realizar a análise de modelos parciais, com isso, pode efetuar análises incrementais
dos modelos enquanto são criados e fornecer resultados instantâneos para o usuário.

\end{itemize}
